\tableofcontents
\newpage

\section{Online-Präsenz}
%\label{sec:first}

\subsection{Definition}

Unter Online-Präsenz versteht man die Internetauftritte des Unternehmens in Form von Webseiten, Social Media, Angeboten auf Job-Portalen und hauseigener Software in App Stores. Die Vielfalt ist heute sehr groß und stetig wachsend. Ein Beispiel verschiedener Möglichkeiten einer Online-Präsenz ist in Tabelle \ref{tab:beispiele} und \ref{tab:beispiele2} zu sehen. %auf Seite \pageref{tab:beispiele}

\begin{table}[H]
\centering
\caption{Beispiele einer Online-Präsenz}
\label{tab:beispiele}
\begin{tabular}{ | p{4cm} | p{3cm} | p{6cm} | }
\hline
Form & Beispiel & Inhalt / Bedeutung \\ 
\hline
Eigene Webseite &
.de \newline
.com \newline 
.net \newline
.org \newline
Domain
&
Informationen zum Unternehmen, \newline
Neuigkeiten, \newline
Pressemitteilungen, \newline
Dienstleistungsangebote, \newline
Standorte, \newline
Öffnungszeiten, \newline
Jobangebote, \newline
Werte und aktuelle Projekte, \newline
Kontaktmöglichkeiten \\
\hline
Social Media \newline (soziale Medien) &
Facebook \newline
Instagram \newline
Twitter \newline
YouTube
&
Neuigkeiten \& Aktuelles, \newline
Jobangebote, \newline
Reaktionen zu aktuellen Ereignissen, \newline
Shorts in Form von Bilder und Videos, \newline
Internes, \newline
Feedback / Beschwerden
\\ \hline
Job Portale & 
Stepstone \newline
Monster \newline
LinkedIn \newline
\href{https://www.kununu.com/}{Kununu} \newline
&Aktuelle Stellenausschreibungen, \newline
Bewertung
\\ \hline
\end{tabular}
\end{table}

\begin{table}[H]
\centering
\caption{Fortsetzung Beispiele einer Online-Präsenz}
\label{tab:beispiele2}
\begin{tabular}{ | p{4cm} | p{3cm} | p{6cm} | }
\hline
Form & Beispiel & Inhalt / Bedeutung \\ 
\hline
Produktplatzierung auf Marktplätze &
Amazon \newline
Ebay \newline
\href{https://store.steampowered.com/}{Steam} \newline
\href{https://itch.io/}{Itch} \newline
\href{https://www.gog.com/}{Gog}
& Zentrale Anlaufstelle für Online-Einkauf, \newline
Keine Eigenentwicklung nötig
\\ \hline
Mobile App &
&
Schnellerer Zugriff auf wichtige Funktionen, \newline
Push Notifications für sofortige Benachrichtigung
\\ \hline
Online-Kartendienst &
Google Maps\newline
Apple Maps \newline
OpenStreetMap &
Genauer Standort, \newline
Für Umkreissuche, \newline
Öffnungszeiten, \newline
Bewertung
\\ \hline
Suchmaschine & Google\newline Bing \newline Yahoo \newline Ecosia &
Link zur Domain, \newline
Gutes Ranking erhöht die Besuche bei Eingabe bestimmter Begriffe
\\ \hline
\end{tabular}
\end{table}

\subsection{Bedeutung}
In der heutigen Welt ist die Online-Präsenz notwendig, da inzwischen weltweit 93 \% das Internet dauerhaft nutzen \cite{internetnutzer}. In Deutschland liegt die tägliche Onlinezeit der Altersgruppe von 30- bis 49-Jährigen bereits bei 258 Minuten. Schaut man nur auf die jüngere Generation der 14- bis 29-Jährigen, sind es sogar 344 Minuten \cite{nutzungsdauer}.

Auch bei der Jobsuche nutzen bereits 87 \% das Internet, um nach potenziellen Arbeitgebern zu suchen. Konventionelle Medien wie Inserate in Zeitungen wird von der Generation bis 40 Jahr mit 16,6 \% kaum noch genutzt \cite{stepstone}.

Je nach Dienstleistung oder Verkaufsware macht das Listen eines Produktes in einer der größeren Marktplätze (Amazon oder Ebay) ebenfalls Sinn. Schon im Jahre 2019 hat man festgestellt, dass 80 \% der Deutschen in den vergangenen 12 Monate auf Amazon eingekauft haben \cite{amazon}.


\subsection{Bewertung}
Die Bewertung der Konkurrenz hilft, den eigenen Internetauftritt zu planen und es besser zu machen. Man erkennt auch, ob eine Neugründung sinnvoll ist. Je nach Art des Unternehmens unterscheiden sich die Prioritäten der jeweiligen Möglichkeiten der Präsenz, was Einfluss auf die Empfehlung hat.

Als Beispiel ist es für ein kleines Indie Gaming Studio weniger interessant eine eigene Internetseite zu haben, sondern eher eine aussagekräftige Produktseite mit veröffentlichten Titeln auf einer der bekannten Marktplätze wie Steam oder Itch.
\section{Arten der Online-Präsenz}

Ausgehend von Tabelle \ref{tab:beispiele} und \ref{tab:beispiele2} wird hier auf die einzelnen Punkte näher eingegangen. Eine kurze Übersicht der wichtigsten Punkte hilft, eine schnelle Bewertung zu treffen.

\subsection{Eigene Webseite}

\subsubsection*{Bewertung}
Eine eigene Webseite zeugt von Professionalität. Im Vergleich zu Social Media hat das Unternehmen vollständige Kontrolle über den Inhalt. Die Bewertung setzt sich aus folgenden Punkten zusammen:

\begin{itemize}
\item Ist die Bedienung intuitiv?\\
Der erste Eindruck zählt. Sind Punkte schwierig zu finden oder behindert durch schlecht platzierte Cookie Banner und Werbung werden Besucher abgeschreckt und verlieren das Interesse die Webseite zu benutzen.
\item Ist die Ladezeit in Ordnung?\\
40 \% verlassen die Webseite falls das Laden mehr als 3 Sekunden andauert \cite{ladezeit}.
\item Kann ein Screenreader die Webseite lesen (Sehbehinderung)?
\item Sind Meldungen aktuell?
\item Kann man zwischen verschiedenen Sprachen wechseln?\\
Sehr wichtig wenn man ein weltweites Publikum ansprechen will. Icon zur Sprachauswahl sollte auf Hauptseite zu finden sein.
\item Wie sieht die Webseite mit verschiedenen Endgeräten aus (Handy, Tablet, PC)?\\
Responsive Design erhöht das Ranking bei Suchmaschinen. 60,04 \% aller Aufrufe online kommen inzwischen von mobilen Endgeräten \cite{mobileDevices}.
\item Gibt es eine Produktliste und Preise?
\item Wie einfach gestaltet sich die Kontaktaufnahme (Formular, E-Mail oder telefonisch)? Wie schnell erfolgt eine Rückmeldung?
\end{itemize}

Ausgehend von diesen Punkten wird die erste Einschätzung der Konkurrenz gemacht. Werden andere Plattformen wie Online-Shops oder Social-Media Kanäle verlinkt kann man davon ausgehen, dass das Unternehmen online stark aufgestellt ist. Werden Produkte mit Preise gelistet, kann direkt mit dem eigenen Produkt verglichen werden.

\subsubsection*{Empfehlung}

Eine eigene Internetpräsenz mit Informationen zum Unternehmen ist immer sinnvoll. Die Kosten sind überschaubar, da eigentlich nur der initiale Aufbau der Seite mit hohen Kosten verbunden ist. Die Pflege danach ist weniger aufwendig, da Inhalte statisch sind und nur bei Bedarf aktualisiert werden. Monatlichen Kosten für Webspace und Domainnamen sind gering.

\subsection{Social Media}

\subsubsection*{Bewertung}

Mehr als ein Drittel (38 \%) kann sich ein Leben ohne Social Media nicht mehr vorstellen. Bei der jüngeren Generation unter 30 Jahren ist der Anteil sogar noch größer \cite{socialmedia}. Umso wichtiger ist der Auftritt in den sozialen Medien. Hier gilt es nicht nur regelmäßig neue Inhalte zu kreieren, sondern auch auf Kundenreaktionen und Aktuelles zeitnah Stellung zu nehmen. Soziale Netzwerke werden oft genutzt, um auf Probleme aufmerksam zu machen, die über den normalen Beschwerdeweg oftmals kein Gehör finden. So musste zum Beispiel der Hersteller von Böhme Fruchtkaramellen den Stehbeutel aus dem Programm nehmen, da über Social Media Beschwerden über das neue Design eingingen \cite{fruchtkaramellen}.

Wenn die Konkurrenz viele verschiedene Social-Media Plattformen unterstützt, ist die öffentliche Arbeit dahinter sehr aktiv. Auf folgende Punkte muss geachtet werden:

\begin{itemize}
\item Auf wie vielen Plattformen ist die Konkurrenz aktiv?
\item Wie viele Follower hat das Unternehmen?\\
Diese Zahlen müssen nicht immer den Erfolg spiegeln. Follower können auch gekauft sein und haben nicht immer eine Korrelation zum Umsatz.
\item Gibt es aktuelle Meldungen und wird regelmäßig neuer Content gepostet?\\
Mit die wichtigste Arbeit, um Follower zu gewinnen und dem Social Media Gedanken Gerecht zu werden.
\item Reagiert das Unternehmen auf Kommentare / Beschwerden?
\item Gibt es bezahlte oder nicht bezahlte Influencer, die auf das Produkt aufmerksam machen?
\end{itemize}

Selbst wenn das Unternehmen gut vertreten ist, kann es ein negatives Bild haben. Es gilt auf Schwachstellen des Unternehmens zu schauen. Über welche Dinge haben sich Personen in der Vergangenheit beschwert? Inhalte dazu können nur schwer gelöscht werden und zeigen, was man bei dem eigenen Auftritt besser machen kann.

\subsubsection*{Empfehlung}

Der Social Media Auftritt mag gut geplant sein. Die initiale Anmeldung ist schnell erledigt, die Pflege ist aber zeitaufwendig und teuer. Je nach Branche ist es dennoch essenziell und wichtig, einen Account zu pflegen. In der Gamingbranche hat es sich inzwischen etabliert, potenzielle Kundschaft mit dem Status der Entwicklung zu informieren, um mehr Interesse zu wecken.


\subsection{Job Portale}
\subsubsection*{Bewertung}
Wie schon erwähnt, werden Jobportale von 87 \% aller Erwerbssuchenden genutzt \cite{stepstone}. Man kann also schauen, wie viele offene Stellen die Konkurrenz hat und welche Art von Bewerber gesucht werden. Darauf kann man auf das Wachstum und die Art der Entwicklung schließen.

Um auf das eigene Unternehmen aufmerksam zu machen, gilt es bessere Konditionen für gleiche oder ähnliche Stelle anzubieten. Zusätzlich bieten Bewertungsportale wie Kununu einen Einblick auf das Arbeitsklima der Konkurrenz.
\subsubsection*{Empfehlung}
Ohne ein Inserat in einem Jobportal wird es schwierig, potenzielle Bewerber zu finden. Die Anmeldung ist also Pflicht. Einen Account bei Kununu sollte auch angelegt werden, um auf negative Bewertungen reagieren zu können.


\subsection{Marktplätze}
\subsubsection*{Bewertung}
Je nach Dienstleistung verkauft das Unternehmen Waren an den Endverbraucher. Dazu werden die Produkte oftmals auf großen Marktplätzen wie Amazon und Ebay gelistet. Der Vorteil des Kunden liegt in der Einfachheit des Einkaufes, einen Account bei diesen Marktplätzen hat fast jeder. Zudem genießt man einen unabhängigen Käuferschutz.

Im Bereich von Computerspiele sind Plattformen wie Steam, Itch und Gog beliebt. Folgende Punkte sind zu bewerten:

\begin{itemize}
\item Auf welchen Plattformen sind Produkte gelistet?
\item Wie schaut das Sortiment dazu aus?
\item Gibt es Preisunterschiede zum hauseigenen Shop, falls vorhanden?
\item Wie sehen die Kundenbewertungen aus?
\item Wie hoch ist die Qualität der Darstellung von dem Artikel?
\item Wie sind die Bewertungen der Kunden?
\item Wie reagiert der Verkäufer auf schlechte Bewertungen?
\end{itemize}

Um konkurrenzfähig zu sein sollte man seine Produkte auf denselben Plattformen anbieten und die schlechten Bewertungen der anderen Unternehmen zur eigenen Verbesserung nutzen.

\subsubsection*{Empfehlung}
Bei Produkten außerhalb von Dienstleistungen empfehlenswert. Das Listen auf einem Marktplatz ist trotz zusätzlicher Kosten meist günstiger als die Pflege eines eigenen Webshops.

\subsection{Mobile App}
\subsubsection*{Bewertung}
Eine mobile App kann den Zugang zu Informationen und Diensten vereinfachen. Eine Statistik von 2021 ergab, dass bei 95 \% der Altersgruppe von 14 bis 49 Jahren ein Smartphone nicht mehr wegzudenken ist \cite{smartphonenutzer}. Folgende Punkte sind zu bewerten:

\begin{itemize}
\item Welche App Stores werden unterstützt?
\item Macht eine App zu diesem Geschäftsmodell überhaupt Sinn?
\item Bewertung der Funktionalität.
\item Besteht ein Vorteil beim Benutzen der App?
\item Wie oft erhält man PUSH Notifications und welchen Inhalt haben diese?
\end{itemize}

\subsubsection*{Empfehlung}
Die Entwicklung und Pflege von Apps für verschiedene Plattformen ist kostenaufwendig. Bei manchen Dienstleistungen macht es dennoch Sinn, eine App zu schreiben. Falls das Budget knapp ist, sollte man aber primär auf eine geeignete mobile Version der Webseite achten.


\subsection{Online-Kartendienst}
\subsubsection*{Bewertung}
Online Kartendienste sind vor allem wichtig bei Firmen mit lokaler Präsenz. Aber selbst wenn ein Unternehmen nur online Waren vertreibt, gibt es oftmals ein Bürogebäude, dass auf Karten verlinkt ist. Google Maps und vergleichbare Dienste ermöglichen es Nutzern, diese Unternehmen zu bewerten. Es ist also auch darauf zu achten, dass hier auf schlechte Bewertung eingegangen wird. Folgende Punkte sind zu bewerten:

\begin{itemize}
\item Gibt es lokale Geschäftsstellen? Wenn ja wo?
\item Wie sehen die Bewertungen aus?
\item Gibt es zusätzliche Bilder, die interessant für den Kunden sein könnten?
\end{itemize}
\subsubsection*{Empfehlung}
Sofern ein Gebäude der Firma existiert, sollte man sich als Eigentümer bei Google und anderen Anbietern eintragen lassen. So kann man die Öffnungszeiten anpassen und auf negative Kommentare reagieren. Eigene Fotos von dem Geschäft und den Produkten kann das Interesse erhöhen.

\subsection{Suchmaschine}
\subsubsection*{Bewertung}
Selbst mit dem Aufstieg von zentralen Social-Media Plattformen spielen Suchmaschinen noch eine große Rolle. Jeder Nutzer benutzt 3 bis 4-mal pro Tag eine Suchmaschine \cite{google}. Die Platzierung der Webseite ist wichtig. Nur noch 0,63 \% klicken auf die Ergebnisse der zweiten Seite \cite{google}. Folgende Punkte sind zu bewerten:

\begin{itemize}
\item Ist die Konkurrenz mit Stichworten zum Produkt über eine Suche auf den ersten Seiten zu finden?
\item Befindet sich diese Seite vor dem eigenen Unternehmen?
\item Findet man einen Eintrag bei Eingabe des Firmennamen?
\item Wird AdSense oder eine ähnliche Werbestrategie benutzt? %\href{https://adsense.google.com/start/}{\textbf{AdSense}}
\end{itemize}

\subsubsection*{Empfehlung}
Besitzt man eine eigene Internetpräsenz mit entsprechender Domain, empfiehlt es sich diese auch für Suchmaschinen zu optimieren. Um das Ranking zu verbessern, gibt es verschiedene Möglichkeiten. Google achtet zum Beispiel auf folgendes:
\begin{itemize}
\item Metadaten
\item Schnelle Ladezeiten
\item Responsive Design (besuchbar mit allen Endgeräten)
\item Verknüpfungen mit Social Media
\item Aktualität der Internetseite
\end{itemize}
\section{Fazit}

Die Art des eigenen Unternehmens entscheidet, welche Art von Präsenz für die Zielgruppe am wichtigsten ist. Der Leser muss also für sich selbst entscheiden, wie viel Zeit und auf welchen Plattformen er die Konkurrenz herausfordern will. Bei einer Neugründung sind die finanziellen Mittel meistens begrenzt, was eine kluge Abwegung besonders wichtig macht.

Zur besseren Einordnung zeigt die Tabelle \ref{tab:kosten} einen einfachen Vergleich.

%\begin{center}
\begin{table}[H]
\centering
\caption{Vergleich der Möglichkeiten}
\label{tab:kosten}
\begin{tabular}{ |l|c|c| }
\hline
Art & Kosten & Relevanz \\ \hline
Webseite & + & + \\ \hline
Social Media & ++ & ++ \\ \hline
Job Portale & + & +++ \\ \hline
Marktplätze & + & +++ \\ \hline
Mobile App & ++ & ++ \\ \hline
Online Kartendienst & + & + \\ \hline
Suchmaschine & ++ & ++ \\ \hline
\end{tabular}
\end{table}
